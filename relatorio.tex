% Modelo da Raquel - Este modelo foi baseado em: modelo-ufpr.tex,v 1.1 2003/06/30 
% incluir o pacote estilo.
% <mariane.raquel@ymail.com>

\documentclass[a4paper,12pt, normaltoc, pnumplain, normalfigtabnum]{style/abntr}
\usepackage[brazil]{babel}
\usepackage[utf8]{inputenc}
\usepackage[T1]{fontenc}
\usepackage{indentfirst}
\usepackage{graphicx}	
\usepackage[left=3cm,right=3cm,top=3cm,bottom=3cm, pdftex]{geometry}
\usepackage{url}
\usepackage{fancyhdr}
\usepackage[font=small,labelfont=bf]{caption}
\usepackage{style/estilo}

 % ***** Início do Documento ******
\begin{document}

% --------- Capa ---------- %
\begin{titlepage}
% Logo e Universidade 
\begin{minipage}{0.2\textwidth}
\begin{flushleft} \includegraphics[scale=0.7]{figuras/logo.png}  \end{flushleft}
\end{minipage}
\hfill
\begin{minipage}{0.7\textwidth}  \begin{flushleft} \begin{center} 
Centro Federal de Educação Tecnológica de Minas Gerais \\
Engenharia de Computação
\end{center} \end{flushleft} \end{minipage}

\begin{center}
\vfill
\begin{Large}
\textbf{ \textsc{Contexto Social e Profissional da Engenharia de Computação: \\  A moral e o esclarecimento}} \\[7cm]            
\end{Large}
% Aluno e Orientador
Mariane Raquel Silva Gonçalves	 \\[3cm]
% Data
Belo Horizonte \\ Abril de 2012 \\ 
\end{center}
\end{titlepage}
% ------- Fim Capa -------- %


% -------  Conteúdo -------- % 

\section*{A moral e o esclarecimento}



% ------ Fim Conteúdo ------ % 
\newpage

\end{document}
